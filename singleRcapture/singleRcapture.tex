\documentclass[
]{jss}

%% recommended packages
\usepackage{orcidlink,thumbpdf,lmodern}

\usepackage[utf8]{inputenc}

\author{
Piotr Chrostowski\\Adam Mickiewicz University \And Maciej
Beręsewicz~\orcidlink{0000-0002-8281-4301}\\Poznań University of
Economics\\
and Bisiness \AND
}
\title{Single-Source Capture-Recapture Models With \pkg{singleRcapture}
and \pkg{singleRcaptureExtra}}

\Plainauthor{Piotr Chrostowski, Maciej Beręsewicz}
\Plaintitle{Single-Source Capture-Recapture Models With singleRcapture
and singleRcaptureExtra}
\Shorttitle{\pkg{singleRcapture}: Single-Source Capture-Recapture
Models}


\Abstract{
Estimating population size is an important issue in official statistics,
social sciences and natural sciences. One way to approach this problem
is to use capture-recapture methods, which can be classified according
to the number of sources used, the main distinction being between
methods based on one source and those based on two or more sources. In
this presentation we will introduce the \pkg{singleRcapture} R package
for fitting SSCR models. The package implements state-of-the-art models
as well as some new models proposed by the authors (e.g.~extensions of
zero-truncated one-inflated and one-inflated zero-truncated models). The
software is intended for users interested in estimating the size of
populations, particularly those that are difficult to reach or for which
information is available from only one source and dual/multiple system
estimation cannot be used.
}

\Keywords{population size estimation, truncated distributuons, count
regression models, \proglang{R}}
\Plainkeywords{population size estimation, truncated
distributuons, count regression models, R}

%% publication information
%% \Volume{50}
%% \Issue{9}
%% \Month{June}
%% \Year{2012}
%% \Submitdate{}
%% \Acceptdate{2012-06-04}

\Address{
    Piotr Chrostowski\\
    Adam Mickiewicz University\\
    Faculty of Mathetmatics and Informatics\\
Wieniawskiego 1\\
61-712 Poznań, Poland\\
  E-mail: \email{piochl@st.amu.edu.pl}\\
  URL: \url{https://github.com/Kertoo}\\~\\
      Maciej Beręsewicz\\
    Statistical Office in Poznań\\
    Poznań University of Economics and Business\\
Department of Statistics\\
Institute of Informatics and Quantitative Economics\\
Al. Niepodległosci 10\\
61-875 Poznań, Poland\\
  E-mail: \email{maciej.beresewicz@ue.poznan.pl}\\
  
  }


% tightlist command for lists without linebreak
\providecommand{\tightlist}{%
  \setlength{\itemsep}{0pt}\setlength{\parskip}{0pt}}




\usepackage{amsmath}

\begin{document}



\hypertarget{introduction}{%
\section{Introduction}\label{introduction}}

This template demonstrates some of the basic LaTeX that you need to know
to create a JSS article.

\hypertarget{code-formatting}{%
\subsection{Code formatting}\label{code-formatting}}

In general, don't use Markdown, but use the more precise LaTeX commands
instead:

\begin{itemize}
\item
  \proglang{Java}
\item
  \pkg{plyr}
\end{itemize}

One exception is inline code, which can be written inside a pair of
backticks (i.e., using the Markdown syntax).

If you want to use LaTeX commands in headers, you need to provide a
\texttt{short-title} attribute. You can also provide a custom identifier
if necessary. See the header of Section \ref{r-code} for example.

\section[R code]{\proglang{R} code}\label{r-code}

Can be inserted in regular R markdown blocks.

\begin{CodeChunk}
\begin{CodeInput}
R> x <- 1:10
R> x
\end{CodeInput}
\begin{CodeOutput}
 [1]  1  2  3  4  5  6  7  8  9 10
\end{CodeOutput}
\end{CodeChunk}

\subsection[Features specific to rticles]{Features specific to
\pkg{rticles}}\label{features-specific-to}

\begin{itemize}
\tightlist
\item
  Adding short titles to section headers is a feature specific to
  \pkg{rticles} (implemented via a Pandoc Lua filter). This feature is
  currently not supported by Pandoc and we will update this template if
  \href{https://github.com/jgm/pandoc/issues/4409}{it is officially
  supported in the future}.
\item
  Using the \texttt{\textbackslash{}AND} syntax in the \texttt{author}
  field to add authors on a new line. This is a specific to the
  \texttt{rticles::jss\_article} format.
\end{itemize}




\end{document}
